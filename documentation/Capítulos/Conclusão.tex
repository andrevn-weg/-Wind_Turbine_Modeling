
\par Este trabalho tem como objetivo desenvolver um sistema de simulação para turbinas eólicas utilizando MATLAB/SIMULINK®, focando na modelagem e análise do comportamento de aerogeradores, especialmente o modelo TE24, sob diferentes condições de vento. O desenvolvimento desse simulador permitira a avaliação detalhada dos componentes da turbina, como pás, rotor, gerador e sistemas de controle, contribuindo para uma melhor compreensão de como otimizar a eficiência energética dos aerogeradores.

\par A metodologia empregada incluiu a integração de dados de vento reais da cidade de Cachoeira do Sul, possibilitando a criação de cenários de simulação realistas. A validação dos modelos teóricos com dados empíricos demonstrou a precisão das simulações realizadas, destacando a relevância do uso de software como MATLAB/SIMULINK® para estudos na área de energia eólica.

\par Os resultados obtidos reforçam a importância do desenvolvimento da plataforma de simulação na previsão de desempenho. O desenvolvimento e a validação de modelos precisos são essenciais para o avanço das tecnologias de conversão de energia eólica, contribuindo para a inserção eficiente de aerogeradores na matriz energética brasileira.

\par Além disso, este trabalho evidencia o potencial de crescimento da energia eólica no Brasil, destacando a relevância de investimentos contínuos em pesquisa e desenvolvimento na área. A utilização de fontes renováveis como a energia eólica é crucial para a diversificação da matriz energética e para a redução dos impactos ambientais associados à geração de eletricidade.

\par Em suma, a pesquisa alcançou seus objetivos para desenvolvimento do TCC 1, assim, tendo um primeiro modelo de perfil eólico e  sugerindo a integração de sistemas de supervisão e controle mais avançados, bem como a análise de diferentes regimes operacionais para turbinas eólicas. A continuidade desse tipo de pesquisa é fundamental para o desenvolvimento sustentável do setor energético.