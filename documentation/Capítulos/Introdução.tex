\par A crescente necessidade de geração de energia elétrica, impulsionada pelo aumento da população e pela expansão econômica, tem colocado em evidência a urgência de se encontrar soluções energéticas que sejam ao mesmo tempo eficientes e sustentáveis. Nos últimos anos, a busca por alternativas que reduzam a dependência dos combustíveis fósseis e minimizem os impactos ambientais tem se tornado uma prioridade global. Nesse contexto, fontes de energia renovável, como a energia eólica, emergem como opções viáveis e necessárias para atender à demanda crescente por eletricidade de forma ambientalmente responsável.
\par De acordo com a \citeonline{iea2024}, o consumo mundial de energia elétrica cresceu cerca de 3,1\% ao ano na última década, refletindo o aumento das atividades industriais e a melhoria dos padrões de vida. No Brasil, a \citeonline{epe2024} destaca que, entre 2010 e 2020, o consumo de energia elétrica aumentou em média 4,2\% ao ano, influenciado tanto pelo crescimento econômico quanto pelo aumento da população urbana. 
\par Para atender ao crescente consumo de energia elétrica, muitos países têm recorrido ao aumento da geração a partir de combustíveis fósseis, como carvão, petróleo e gás natural. Esse incremento, porém, tem causado uma escalada significativa nas emissões de gases de efeito estufa, agravando o aquecimento global. Segundo \citeonline{calijuri2013}, "a matriz elétrica mundial é predominantemente fóssil, e os esforços para reduzir a dependência desses combustíveis de alto impacto ambiental são cruciais para mitigar as mudanças climáticas.". Este cenário evidencia a necessidade urgente de diversificar a matriz energética. Em resposta a isso, a comunidade internacional tem intensificado a busca por alternativas mais limpas, como a energia eólica, e promovido campanhas e acordos para incentivar o uso de fontes renováveis. Iniciativas como o Protocolo de Kyoto \cite{kyoto2024} e o Acordo de Paris \cite{paris2024}, que introduzem mecanismos de crédito de carbono, têm sido fundamentais para estimular investimentos em tecnologias sustentáveis e reduzir as emissões globais de carbono.
\par O Brasil tem se destacado na implementação de energias renováveis, com uma matriz energética que é considerada uma das mais limpas do mundo. De acordo com o \citeonline{ben2020}, cerca de 83\% da eletricidade gerada no país provém de fontes renováveis, com a energia hídrica sendo a principal contribuição. Nos últimos anos, a energia eólica tem ganhado uma importância crescente no cenário brasileiro. Em 2020, a energia eólica foi responsável por aproximadamente 10\% da geração elétrica do Brasil, com uma capacidade instalada que ultrapassou 16 gigawatts (GW), segundo a \citeonline{abeeolica2020}. Além disso, o país possui um vasto potencial de crescimento nesse setor. Estudos indicam que o Brasil tem capacidade para instalar mais de 500 GW de energia eólica, especialmente nas regiões Nordeste e Sul, onde os ventos são mais favoráveis. Este potencial coloca o Brasil em uma posição estratégica para expandir ainda mais sua capacidade de geração eólica, contribuindo significativamente para a diversificação de sua matriz energética e para a mitigação dos impactos ambientais associados à geração de energia.

\par A expansão do setor de energia eólica no Brasil tem sido notável, impulsionada por investimentos substanciais tanto do setor público quanto do privado. Segundo a \citeonline{abeeolica2024}, os investimentos acumulados em energia eólica no Brasil desde 2009 já superam a marca de R\$ 100 bilhões. Este influxo de capital tem permitido a construção de novos parques eólicos e a modernização de infraestruturas existentes, aumentando significativamente a capacidade de geração. Em 2020, foram investidos cerca de R\$ 13 bilhões em novos projetos, com previsão de investimentos adicionais de R\$ 62 bilhões até 2024, conforme relatado pelo \citeonline{bndes2024} e \citeonline{estadao2024}. Este cenário de crescimento contínuo reflete a confiança dos investidores na energia eólica como uma fonte sustentável e lucrativa, além de demonstrar o compromisso do Brasil em se tornar um líder global na geração de energia renovável.
\par Portanto, é evidente que o campo de geração eólica ainda está em expansão. 
Dado esse aumento, torna-se cada vez mais necessário  estudos para prever e reduzir a  ocorrência de problemas diversos no controle de turbinas. Desta forma, o presente trabalho busca desenvolver uma plataforma de simulação de uma turbina eólica, afim de colaborar com estudos no setor e buscar a otimização da qualidade de energia produzida do vento, com o estudo de técnicas de modos de operação, velocidade e limitação de potência.




%\section{Motivação}

\section{Objetivo geral}

    \par O objetivo do trabalho é desenvolver um simulador de turbina eólica de velocidade variável. Além disso, busca-se desenvolver um sistema de supervisão, possibilitando testar a turbina em diferentes regimes de vento e pontos de operação.

%\textcolor{red}{(SUGESTÃO: O objetivo do trabalho é desenvolver um simulador de turbina eólica de velocidade variável. Além disso, busca-se desenvolver um sistema de supervisão, possibilitando testar a turbina em diferentes regimes de vento e pontos de operação.)}

\section{Objetivos específicos}
%\textcolor{red}{(Eu acredito que seja melhor dividir em mais pontos e nao colocar nada de experimental.Por exemplo: 1. Revisão da literatura sobre sistemas de conversão eólica; 2. Obtenção de perfil de vento médio de Cachoeira do sul; 3. obtenção de série temporal do vento, com turbulência e rajadas de vento; 4. Desenvolvimento do sistema de controle para uma turbina eólica de 20 kW, desde velocidade minima de cut-in, técnica de mpppt, limitação de potência e cutt-off. 5; Implementar uma plataforma de simulação que possibilite a modelagem e análise de todos os componentes de uma turbina eólica (pode até comentar de ser algo em python); 6; Desenvolvimento de um sistema de supervisão (python); 7. Por fim, até pode comentar que pretendo avaliar o desempenho do sistema de emulação em diferentes cenários operacionais)}



\begin{enumerate}
    \item Revisão da literatura sobre sistemas de conversão eólica.

    \item Obtenção do perfil médio de vento em Cachoeira do Sul.
    
    \item Aquisição da série temporal do vento, incluindo turbulência e rajadas de vento.

    \item Desenvolvimento de um sistema de controle para uma turbina eólica de 20 kW, abrangendo desde a velocidade mínima de \textit{cut-in}, técnica de MPPT, limitação de potência e \textit{cut-off}.

    \item Implementação de uma plataforma de simulação que permita a modelagem e análise de todos os componentes de uma turbina eólica (com possibilidade de uso de Python).

    \item Desenvolvimento de um sistema de supervisão utilizando Python.

    \item Avaliação do desempenho do sistema de emulação em diferentes cenários operacionais.
    
\end{enumerate}

\section{Organização do Trabalho}
\par O trabalho está organizado da seguinte maneira:
\par O capítulo 2 apresenta uma revisão da literatura sobre os principais conceitos relacionados ao estudo. Neste capítulo é obtido e desenvolvido a análise dos ventos, perfil de vento, potência gerada pelos ventos, série temporal do vento e o mapa eólico do Brasil. Também são discutidos os modelos matemáticos e físicos que fundamentam a simulação de turbinas eólicas.
\par No capítulo 3 aborda os conceitos básicos de energia eólica e princípios fundamentais das turbinas eólicas. São descritos os componentes de um aerogerador de eixo horizontal, com destaque para a turbina eólica TE24, e as características técnicas e construtivas relevantes para o desenvolvimento do simulador.
\par No capítulo 4 é detalhada a metodologia utilizada para o desenvolvimento do sistema de simulação de turbinas eólicas. Inclui a descrição das ferramentas e técnicas empregadas, como o software MATLAB/SIMULINK, bem como os critérios e procedimentos para a modelagem e simulação das séries temporais do vento e do sistema de emulação do aerogerador.
\par No capítulo 5, são expostas as conclusões gerais do estudo.

